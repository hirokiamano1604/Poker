\documentclass[a4paper, 11pt]{ltjsarticle}
% --- 基本設定 ---
% \usepackage{luatexja-fontspec} % lualatexで日本語フォント指定
% Macのヒラギノフォントを指定 (存在しない場合は他のフォントに変更してください)
% \setmainjfont[BoldFont={Hiragino Sans W6}, ItalicFont={Hiragino Mincho ProN W3}]{Hiragino Sans W3}
% \setsansjfont[BoldFont={Hiragino Sans W6}]{Hiragino Sans W3}
% \setmonojfont{Osaka-等幅} % Osakaフォントがない場合は他の等幅フォントを指定

% \usepackage[haranoaji]{luatexja-preset} % Harano Aji Fonts (Noto Fonts 由来) を使う場合
\usepackage[hiragino-pro,deluxe]{luatexja-preset} % ヒラギノフォントを使う場合 (より詳細な設定)

% \usepackage[T1]{fontenc} % LuaLaTeX + fontspecでは通常不要
\usepackage{amsmath, amssymb, mathtools, amsthm} % 数式関連
\usepackage{graphicx} % 画像挿入
\usepackage{hyperref} % ハイパーリンク
\usepackage{listings} % ソースコード表示
\usepackage{enumitem} % リスト環境のカスタマイズ
\usepackage{graphicx}
\usepackage{mathrsfs}
\usepackage{romanbar}

% --- Beamerテーマと配色 (記事形式では不要なため削除・コメントアウト) ---
% \usetheme{Madrid}
% \usecolortheme{default}

% カスタムカラー (tcolorboxなどで使用するため残す)
% \definecolor{myblue}{RGB}{0,110,180}
% \definecolor{mygreen}{RGB}{0,150,100}
% \definecolor{myorange}{RGB}{230,120,20}
% \definecolor{mygray}{RGB}{240,240,240}

% Beamerの配色を一部上書き (記事形式では不要なため削除・コメントアウト)
% \setbeamercolor{palette primary}{bg=myblue,fg=white}
% ... (他のsetbeamercolorも同様に削除) ...
% \setbeamercolor{block body}{bg=mygray}

% --- tcolorboxの設定 (スタイリッシュな囲み) ---
\usepackage[most]{tcolorbox}

%================================================================
% 定理・定義・命題などのカスタムボックス環境
%================================================================

% --- 必要なパッケージ(プリアンブルにあることを確認)---
% \usepackage{amsthm}
% \usepackage{amssymb}
% \usepackage[most]{tcolorbox}
% \tcbuselibrary{theorems} % \newtcbtheorem を使うために必要


% --- カスタムカラー定義 ---
\definecolor{myblue}{RGB}{0,110,180}
\definecolor{mygreen}{RGB}{0,150,100}
\definecolor{myorange}{RGB}{230,120,20}
\definecolor{mypurple}{RGB}{120, 80, 190}
\definecolor{mycautionred}{RGB}{192,0,0}
\definecolor{mycautionyellow}{RGB}{255,242,204}


% --- 自動採番される環境の定義 ---

% まず「定理」を基準として定義。番号はセクションごとにリセット
\newtcbtheorem[number within=subsection]{mytheorem}{定理}{
  breakable,
  colback=mygreen!5!white,
  colframe=mygreen!75!black,
  colbacktitle=mygreen!75!black,
  coltitle=white,
  fonttitle=\bfseries
}{thm}

% 定義・命題・補題は、「定理」と同じ番号カウンターを共有する
\newtcbtheorem[use counter from=mytheorem]{mydefinition}{定義}{
  breakable,
  colback=myblue!5!white,
  colframe=myblue!75!black,
  colbacktitle=myblue!75!black,
  coltitle=white,
  fonttitle=\bfseries
}{def}

\newtcbtheorem[use counter from=mytheorem]{myproposition}{命題}{
  breakable,
  colback=mypurple!5!white,
  colframe=mypurple!75!black,
  colbacktitle=mypurple!75!black,
  coltitle=white,
  fonttitle=\bfseries
}{prop}

\newtcbtheorem[use counter from=mytheorem]{mylemma}{補題}{
  breakable,
  width=0.9\textwidth,  % 幅を90%に
  center,               % 中央揃え
  colback=mygreen!5!white,
  colframe=mygreen!75!black,
  colbacktitle=mygreen!75!black,
  coltitle=white,
  fonttitle=\bfseries
}{lem}


% --- 手動でタイトルを指定する、または番号なしの環境 ---

% 例題用のスタイル(番号やタイトルは手動で指定)
\newtcolorbox{myexample}[2][]{
  breakable,
  colback=myorange!5!white,
  colframe=myorange!75!black,
  colbacktitle=myorange!75!black,
  coltitle=white,
  fonttitle=\bfseries,
  title={例題: #2},
  #1
}

% 注意書き用のスタイル(タイトルは「注意」で固定、番号なし)
\newtcolorbox{mycaution}[1][]{
  breakable,
  colback=mycautionyellow,
  colframe=mycautionred,
  colbacktitle=mycautionred,
  coltitle=white,
  fonttitle=\bfseries,
  title={注意},
  #1
}


\newcommand{\bb}[1]{\mathbb{#1}}
\newcommand{\mcal}[1]{\mathcal{#1}}
\newcommand{\st}{\quad \text{s.t.} \quad}
\newcommand{\prsp}{(\Omega, \mcal{F}, \bb{P})}
\DeclareMathOperator*{\argmin}{arg\,min}
\DeclareMathOperator*{\argmax}{arg\,max}

% --- 段落インデント設定 ---
\setlength{\parindent}{0pt} % デフォルトはインデントなし

\newlength{\manualparindent}
\setlength{\manualparindent}{1\zw} % ← ここが重要(1zw ではなく 1\zw)


\newcommand{\pindent}{\hspace*{\manualparindent}} % 手動で入れるインデント

\title{Introduction to Asset-Lianility Management}
\date{2026/1/1}

\begin{document}
\maketitle

\section{From Asset Management to Asset-Liability Management}
There was a PENSION FUND CRISIS.
The S\&P 500 DB Pension Plans fell enormouls between Dec 1999 and May 2003.
At the Dex 1999, it had a net surplus of \$ 239 billion, and at the end of the period the surplus turned into a net deficit of \$ 252 billion.
It was in the end of Tech bubble. NASDAQ index also declined grately in that period.

The simmilar thing happend again in 2008.

In asset-liability management, the asset value against the liability value matters.
In other words,
\[F_t = \frac{A_t}{L_t}\]
And surplus is given by a difference between assets and liability values:
\[S_t = A_t - L_t\]

\section{Liability hedging portfolios}
The investers concern mainly about an unexpected increase in the present value of thier liabilities.

A possible solution is called Liability-Hedging portfolios(LHR) or Goal-Hedging portfolio.
There should be a certain cashflow that the liability will take, so that we need to try to make the similar cashflow by adjusting our portfolio.

With respect to the goal, a standard vond is not safe because it only makes a temporary huge cashflow in one period and the payoff of the bond change dramatically depending on the other factors.
The portfolio should be like a retirement bond.

In reality, convinient bond which precisely fit the cashflow unlikely to exist so that we use \textbf{Duration Matching}.
Since duration is the figure of sensitivity against inflation or interest rate change

\section{Liability-driven investing (LDI)}
In order to have a better performance, the one have to expose hisself in to the "Risk".
The hedging is the method to control the risk appropriately to prepare for an unexpected shocks.
Performance-seeking portfolio (PSP) and Liability Hedging Portfolio (LHR) are the one of the strategies which try to tame risk.

The optimal allocation strategy is giben by
\[\max_{w} \bb{E}\left[u\left(\frac{A_T}{L_T}\right)\right]\]
\[\therefore w^* = \frac{\lambda_{\text{PSP}}}{\gamma \sigma_{\text{PSP}}}w^{\text{PSP}} + \beta_{\text{L, LHP}}(1- \frac{1}{\gamma})w^{\text{LHP}}\]
where $A_T$ is an asset value, $L_T$ is a liability value and $u$ is an utility function

In the LDI, they also have \textbf{Greeks}
\begin{enumerate}
    \item $\lambda_{\text{PSP}}$ is an sharpe ratio
    \item $\beta_\text{LHP}$ is the beta, sensitivity, of liability
    \item $\sigma_{\text{PSP}}$ is a volatility of the PSP strategy
    \item $\gamma$ is a risk-aversion
\end{enumerate}

\section{Choosing the policy portfolio}
How do we decide an optimal allocation to PSP and LHP for a given investor?
This should be related to a risk aversion parameter $\gamma$, however it is impossible to capture the figure.
Therefore, we should think of it as free parameter.

In practice, investers have a risk budget, and they can increase the allocation to PSP until the budget run out.
The risk budgets depdn on the stake holders or agents.

There must be a conflict between short-term and long-term perspective, and also they hace a dollar budget.
If you have only a little risk budget, you need to have an enormous dollar budget to get a certain amount of benefit.

To tackle with this dilemma, one way is to uhide the problem by using a higher discount rate
or higher risk premia value, and another way is to get a higher dollar budget or a higher risk budget.

\section{Beyond LDI}
For investers, a well-designed portfolio strategy is the most reliable one.
The LDI paradigm is the one for this wish.
The dilemma between performance and hedging motive is very important thing in constructing portfolio strategy.

After skipping some mathematical procedure, we can get a Investor Welfare as
\[\text{IW} = \frac{\lambda_{\text{PSP}}^2}{ 2 \gamma} + \frac{(1-\gamma)^2}{2 \gamma} \sigma_{\text{L}}^2 \rho_{\text{L,LHP}}^2 + \left(1-\frac{1}{\gamma} \sigma_{\text{L}} \rho_{\text{L,LHP}} \lambda_{\text{PSP}}\right)\]
the first term represent a pure perf contribution(risk premia by PSP), the second term represent a pure hedging contribution(The hedging effect by LHP) and the last term is a cross-contribution per/hedging.

\section{Liability Friendly Equity Portfolio}
What's the definition of "Liability Friendliness"?

One answer is that the equity can provide a cash-flow which matches the one of the liability.
Another aspect is a factor matching focus.

\end{document}