\documentclass[a4paper, 11pt]{ltjsarticle}
% --- 基本設定 ---
% \usepackage{luatexja-fontspec} % lualatexで日本語フォント指定
% Macのヒラギノフォントを指定 (存在しない場合は他のフォントに変更してください)
% \setmainjfont[BoldFont={Hiragino Sans W6}, ItalicFont={Hiragino Mincho ProN W3}]{Hiragino Sans W3}
% \setsansjfont[BoldFont={Hiragino Sans W6}]{Hiragino Sans W3}
% \setmonojfont{Osaka-等幅} % Osakaフォントがない場合は他の等幅フォントを指定

% \usepackage[haranoaji]{luatexja-preset} % Harano Aji Fonts (Noto Fonts 由来) を使う場合
\usepackage[hiragino-pro,deluxe]{luatexja-preset} % ヒラギノフォントを使う場合 (より詳細な設定)

% \usepackage[T1]{fontenc} % LuaLaTeX + fontspecでは通常不要
\usepackage{amsmath, amssymb, mathtools, amsthm} % 数式関連
\usepackage{graphicx} % 画像挿入
\usepackage{hyperref} % ハイパーリンク
\usepackage{listings} % ソースコード表示
\usepackage{enumitem} % リスト環境のカスタマイズ
\usepackage{graphicx}
\usepackage{mathrsfs}
\usepackage{romanbar}

% --- Beamerテーマと配色 (記事形式では不要なため削除・コメントアウト) ---
% \usetheme{Madrid}
% \usecolortheme{default}

% カスタムカラー (tcolorboxなどで使用するため残す)
% \definecolor{myblue}{RGB}{0,110,180}
% \definecolor{mygreen}{RGB}{0,150,100}
% \definecolor{myorange}{RGB}{230,120,20}
% \definecolor{mygray}{RGB}{240,240,240}

% Beamerの配色を一部上書き (記事形式では不要なため削除・コメントアウト)
% \setbeamercolor{palette primary}{bg=myblue,fg=white}
% ... (他のsetbeamercolorも同様に削除) ...
% \setbeamercolor{block body}{bg=mygray}

% --- tcolorboxの設定 (スタイリッシュな囲み) ---
\usepackage[most]{tcolorbox}

%================================================================
% 定理・定義・命題などのカスタムボックス環境
%================================================================

% --- 必要なパッケージ(プリアンブルにあることを確認)---
% \usepackage{amsthm}
% \usepackage{amssymb}
% \usepackage[most]{tcolorbox}
% \tcbuselibrary{theorems} % \newtcbtheorem を使うために必要


% --- カスタムカラー定義 ---
\definecolor{myblue}{RGB}{0,110,180}
\definecolor{mygreen}{RGB}{0,150,100}
\definecolor{myorange}{RGB}{230,120,20}
\definecolor{mypurple}{RGB}{120, 80, 190}
\definecolor{mycautionred}{RGB}{192,0,0}
\definecolor{mycautionyellow}{RGB}{255,242,204}


% --- 自動採番される環境の定義 ---

% まず「定理」を基準として定義。番号はセクションごとにリセット
\newtcbtheorem[number within=subsection]{mytheorem}{定理}{
  breakable,
  colback=mygreen!5!white,
  colframe=mygreen!75!black,
  colbacktitle=mygreen!75!black,
  coltitle=white,
  fonttitle=\bfseries
}{thm}

% 定義・命題・補題は、「定理」と同じ番号カウンターを共有する
\newtcbtheorem[use counter from=mytheorem]{mydefinition}{定義}{
  breakable,
  colback=myblue!5!white,
  colframe=myblue!75!black,
  colbacktitle=myblue!75!black,
  coltitle=white,
  fonttitle=\bfseries
}{def}

\newtcbtheorem[use counter from=mytheorem]{myproposition}{命題}{
  breakable,
  colback=mypurple!5!white,
  colframe=mypurple!75!black,
  colbacktitle=mypurple!75!black,
  coltitle=white,
  fonttitle=\bfseries
}{prop}

\newtcbtheorem[use counter from=mytheorem]{mylemma}{補題}{
  breakable,
  width=0.9\textwidth,  % 幅を90%に
  center,               % 中央揃え
  colback=mygreen!5!white,
  colframe=mygreen!75!black,
  colbacktitle=mygreen!75!black,
  coltitle=white,
  fonttitle=\bfseries
}{lem}


% --- 手動でタイトルを指定する、または番号なしの環境 ---

% 例題用のスタイル(番号やタイトルは手動で指定)
\newtcolorbox{myexample}[2][]{
  breakable,
  colback=myorange!5!white,
  colframe=myorange!75!black,
  colbacktitle=myorange!75!black,
  coltitle=white,
  fonttitle=\bfseries,
  title={例題: #2},
  #1
}

% 注意書き用のスタイル(タイトルは「注意」で固定、番号なし)
\newtcolorbox{mycaution}[1][]{
  breakable,
  colback=mycautionyellow,
  colframe=mycautionred,
  colbacktitle=mycautionred,
  coltitle=white,
  fonttitle=\bfseries,
  title={注意},
  #1
}


\newcommand{\bb}[1]{\mathbb{#1}}
\newcommand{\mcal}[1]{\mathcal{#1}}
\newcommand{\st}{\quad \text{s.t.} \quad}
\newcommand{\prsp}{(\Omega, \mcal{F}, \bb{P})}
\DeclareMathOperator*{\argmin}{arg\,min}
\DeclareMathOperator*{\argmax}{arg\,max}

% --- 段落インデント設定 ---
\setlength{\parindent}{0pt} % デフォルトはインデントなし

\newlength{\manualparindent}
\setlength{\manualparindent}{1\zw} % ← ここが重要(1zw ではなく 1\zw)

\title{Module 1}
\author{Author}
\date{2026/1/1}

\begin{document}

\maketitle

\section{Fundamentals of Returns}
Let $P_t$ be the price of the asset.

The return on this asset is given by
\[R_{t,t+1} = \frac{P_{t+1} - P_t}{P_t}\]

The word "return" in this finance context, you should come up with this word.

\section{Measures of Risk and Reward}
The variance is given by 
\[\sigma_R^{2} = \frac{1}{N} (R_i - \bar{R})^2\]
where $\bar{R}$ is the arithmetric mean of the return.

And the standard diviation is computed by
\[\sigma_R = \sqrt{\frac{1}{N} (R_i - \bar{R})^2}\]

SInce you cannot compare daily data from monthly data, you need to use
\[\sigma_{ann} = \sigma_p \sqrt{p}\]
where p is the number of periods.

For example, to standaralize a daily data with a annual data,
\[\sigma_{ann} = \sigma_{daily} \sqrt{365}\]

To see the excess return, we see the $\textbf{Sharp ratio}$
\[\frac{R_p - R_f}{\sigma_p}\]
where $R_p$ is a return of the portfolio and $\sigma_p$ is a stdev of the portfolio

\section{Measuring Max Drawdown}
Instead of the risk measure, we use \textbf{Max drawdown}

This is the value of the maximum loss from the previous high to a subsequent low.

In order to compute the drawdowns, we will take these steps:
\begin{enumerate}
  \item Construct a wealth index
  \item Look at the prior peak at any point in time
  The drawdown is literally the difference between the prior peak and the current value
  \item We can plot the drawdown by seeing how long does it tak to recover from the drawdowns
\end{enumerate}

The value heavily depends on the frequwncy of observations. In general, the drawdown of daily series data is greater than the one of monthly series data.

\section{Deviations from Normality}
We assume that a asset return is \textbf{Normally} distributed.

It is called "The Gaussian Assumption"

But somtimes they larger changes which contradict to this assumption.

The skewness is given by
\[\frac{\bb{E}\left[(R-\bb{E}[R])^3\right]}{\left[Var(R)\right]^{\frac{3}{2}}}\]
It is basically the measure of a symmetry. If the data is inclined in the right side, the skewness will be positive and vice versa.

The kurtosis s given by
\[\frac{\bb{E}\left[(R-\bb{E}[R])^4\right]}{\left[Var(R)\right]^{2}}\]
It is a measure of a possibility of extreme events.

For the Gaussian distribution, it takes 0 for skewness and 3 for kurtosis.

But in a reality, asset returns are not normally distributed.

\section{Downside risk measure}
Deviation is not always  a bad signal. Only downside deviation is the risk of the asset.

To capture this downside risk, we have a \textbf{Semi-deviation}.

This is the sub-sample of below-average or below-zero returns which is given by
\[\sigma_{semi} = \sqrt{\frac{1}{N} \Sigma_{R_t \leq \bar{R}}^{ } (R_t - \bar{R})^2}\]
where $N$ is the number of returns that fall below the mean. 

Additionlayy, there is a concept which is called as \textbf{Value at Risk(VaR)}.

To think of it, we have to set the things below
\begin{enumerate}
  \item A specified confidence level (Percentage)
  \item A specified holding period
\end{enumerate}
The VaR is the absolute value of worst return in the specified period. 
Suppose the confidence level is $n$.
We need to exclude tha last (1-n)\% from the returns for calculation.

Also, there is \textbf{Conditional VaR} which is given by
\[\text{CVaR} = -\bb{E}[R|R \leq -VaR] = \frac{-\int_{-\infty}^{-VaR} x f_R(x) dx}{F_R(-VaR)}\]
This means the expected loss beyond VaR.
Pay attention to that this CVaR only returns positive number.

\section{Estimating VaR}
There are mainly 4 typrs of calulating VaR.
\begin{enumerate}
  \item Historical(non-parametric)
  \item Variance-Covariance(Parametric Gaussian)
  \item Parametric non Gaussian
  \item Cornish-Fisher (Semi-parametric)
\end{enumerate}

For the first one, we use the histrical data.
The pro is that they don't assume, however, the accuracy heavily depends on the sample periods.

For the second one, by assuming the Gaussian for the returns, and estimate the parameters.
One of the method is Gaussian VaR.
We think of $Z_{\alpha}$ as the $\alpha$-quantile of the standard normal distribution.
For example, $Z_{\alpha} = -1.65$at $5\%$

We just need to think of a $Z_\alpha$ which satisfy
\[\int_{-\infty}^{Z_{\alpha}} \frac{1}{\sigma \sqrt{2 \pi}} \exp{\frac{-x^2}{2\sigma^2}}dx = \alpha\]
\[\therefore \bb{P}\left(\frac{R-\mu_P}{\sigma_P} \leq Z_\alpha\right) = \alpha\]
\[\therefore \bb{P}\left(R\leq \mu_P + \sigma_\alpha Z_\alpha\right) = \alpha\]
and we can get:
\[VaR_\alpha = ^(\mu + Z_\alpha \sigma)\]

For the third one, there are a lot of methodology for this.

When you take a parametiric way, you will have a model-risk.
So that we take a Cornish-Fisher VaR, which is a part of semi-parametric approach.
According to the Cornish-Fisher expansion,
\[\tilde{Z_\alpha} = Z_\alpha + \frac{1}{6} (Z_\alpha^2-1) S + \frac{1}{24}(Z_\alpha^3 -3Z_\alpha)(K-3) - \frac{1}{36} (2 Z_\alpha^3-5Z_\alpha)S^2\]
where $\tilde{Z_\alpha}$ have a $\alpha$-quantile of non-Gaussian distribution, S is a skewness and K is a kurtosis.

In many cases, the form of
\[Var_{mod} (1-\alpha) = - (\mu + \tilde{Z_\alpha}\sigma)\]
is taken.




\end{document}